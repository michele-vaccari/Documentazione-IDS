% !TEX encoding = UTF-8
% !TEX TS-program = pdflatex
% !TEX root = ../Tesi.tex
% !TEX spellcheck = it-IT

%*******************************************************
% Introduzione
%*******************************************************
\cleardoublepage
\pdfbookmark{Introduzione}{introduzione}

\chapter*{Introduzione}

BINS (acronimo ricorsivo di "BINS Is Not Shopping") è un applicazione web liberamente disponibile, indicata soprattutto per commercializzare prodotti alimentari. Lo scopo di questo lavoro è di documentare l'applicazione usando l'UML (Unified Model Language).

La documentazione è articolata come segue.

\begin{description}
	\item[{\hyperref[cap:specifiche-progetto]{Il primo capitolo}}]
	contiene le specifiche del progetto descrivendo le caratteristiche e le categorie di utenti dell'applicazione.
	\item[{\hyperref[cap:modello-casi-d'uso]{Il secondo capitolo}}]
	descrive che cosa deve fare il sistema e quali sono gli attori per poterlo far funzionare.
	\item[{\hyperref[cap:modello-del-dominio]{Il terzo capitolo}}]
	descrive quali sono le informazioni che il sistema deve trattare e quali non deve trattare.
	\item[{\hyperref[cap:modello-di-design]{Il quarto capitolo}}]
	descrive nel dettaglio la struttura fisica dell'applicazione.
\end{description}

Per il \emph{Presentation Level} si è usato il framework \emph{Bootstrap} sfruttando al meglio il \emph{responsive design} che viene messo a disposizione dal framework. Per quanto riguarda l'usabilità e il design si è scelto l'approccio \emph{mobile first} ma si è anche prestato attenzione all'uso dell'applicazione in modalità desktop.

Per l'application server si è usato \emph{Tomcat} e si è utilizzato il linguaggio di progammazione Java.
 Per interfacciare l'application server con il database si è utilizzato JDBC.
 
Il DBMS utilizzato è \emph{MySQL}.

Di seguito si riporta il diagramma approssimato relativo allo sviluppo del lavoro in termini di tempo attraverso un diagramma di Gantt: