% !TEX encoding = UTF-8
% !TEX TS-program = pdflatex
% !TEX root = ../Tesi.tex
% !TEX spellcheck = it-IT

%************************************************
\chapter{Specifiche progetto}
\label{cap:specifiche-progetto}
%************************************************

\section{Caratteristiche del sistema}
Il sito \emph{SportFerrara} è una piattaforma per la gestione delle attività sportive attraverso una comunità di utenti.

L'applicazione web ha le seguenti caratteristiche:


\begin{enumerate}
	
	\item
	\label{sf-uadmin}
	Possibilità di creare gli utenti per i due ruoli principali: arbitri e gestori di squadra;
	
	\item
	Possibilità di creare i diversi tornei, di disegnarne lo schema. Ogni torneo ha: nome, descrizione e tipo (all'italiana o ad eliminazione).
	
	\item
	Un torneo è costituito da diversi gironi o fasi, in base alla tipologia di torneo. Un girone ha un nome (ad esempio: girone 1) ed è costituito da un insieme di gare. Una fase ha un nome (ad esempio: semifinale 1) ed è costituita da una sola gara;
	
	\item
	Una gara è descritta da: data e ora, luogo, nomi delle squadre che si sfidano, uno o più arbitri;
	
	\item
	\label{ef-uadmin}
	Per ogni torneo è associata una classifica generale che viene aggiornata sulla base dei referti di gara compilati dagli arbitri;
	
	\item
	\label{sf-ugestore}
	Possibilità per il gestore di una squadra di creare una squadra con una rosa di massimo 36 giocatori. Ogni squadra ha un nome e può avere uno sponsor. Per ogni giocatore il gestore può specificare: nome, cognome, luogo e data di nascita, numero di maglia e foto;
	
	\item
	\label{ef-ugestore}
	Per ogni gara cui la squadra è assegnata, il gestore della squadra deve fornire la formazione della squadra composta da nome e cognome dei giocatori e rispettivi ruoli;
	
	\item
	\label{f-uarbitro}
	A gara terminata l'arbitro dovrà compilare un referto in cui annoterà: l'orario effettivo di inizio e di fine della gara, risultato finale, numero di reti con i rispettivi giocatori che li hanno realizzati,  i giocatori espulsi per ogni squadra e i giocatori ammoniti per ogni squadra;
	
	\item
	\label{f-upubblico}
	Possibilità di selezionare il torneo desiderato e visualizzare:
	
	\begin{itemize}
		\item
		La pagina relativa ad una gara con: nomi delle squadre, formazioni e referti;
		
		\item
		La pagina relativa ad una squadra con la rosa dei giocatori e il calendario delle partite;
		
		\item
		La pagina relativa ad un giocatore con le statistiche di gioco (punti realizzati, espulsioni e ammonizioni) per quel torneo.
	\end{itemize}

\end{enumerate}

\section{Utenti del sistema}
Il sistema prevede che le categorie di utenti sia così rappresentata:

\begin{description}
	
	\item[Amministratori] Possono effettuare i punti dal \ref{sf-uadmin} al \ref{ef-uadmin} compresi.
	
	\item[Gestori di squadra] Possono effettuare i punti dal \ref{sf-ugestore} al \ref{ef-ugestore} compresi.
	
	\item[Arbitri] Possono effettuare solamente il punto \ref{f-uarbitro}.
	
	\item[Utenti pubblici] Possono effettuare solamente il punto \ref{f-upubblico}.
	
\end{description}