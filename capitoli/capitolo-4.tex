% !TEX encoding = UTF-8
% !TEX TS-program = pdflatex
% !TEX root = ../Tesi.tex
% !TEX spellcheck = it-IT

%************************************************
\chapter{Modello di design}
\label{cap:modello-di-design}
%************************************************

\section{Diagramma delle classi}
	Il raffinamento del modello delle classi di dominio avviene attraverso il modello di design suddividendo tutte le classi del progetto in tre categorie:
	\begin{enumerate}
		
		\item
		Classi Boundary;
		
		\item
		Classi Control;
		
		\item
		Classi Entity.
		
	\end{enumerate}
	Di seguito vengono presentate le categorie sopracitate.

	\subsection{Classi Boundary}
	Sono le classi che si interfacciano con l'utente.
	
	Nel caso dell'applicazione web, queste sono le singole pagine JSP che in fase di compilazione vengono convertite in Servlet (classi Java che estendono la classe \texttt{HTTPServlet}).
	
	Si procede con l'elenco delle classi Boundary:
	
	\subsection{Classi Control}
	Sono le classi che validano le interazioni dell'utente e si interfacciano con la base dati.
	
	Nel caso dell'applicazione web, queste sono le classi della \emph{Business Logic}, i \emph{Services} associati e le classi che governano il flusso delle conversazioni.
	
	Si procede con l'elenco dele classi \emph{Control} nel package \texttt{BusinessLogic}:
	
	%
	%
	%
	
	Classi Control nei package di servizio:
	
	%
	%
	%
	
	Classi Control nei packega \texttt{BusinessFow}, il quale si occupa del controllo del flusso. Tali classi usano le classi del package \texttt{BusinessLogic} relative all'oggetto di mappatura del database e ai service interessati dalla conversazione medesima.
	
	Si nota che una conversazione può avere più classi Boundary associate.
	
	\subsection{Classi Entity}
	Sono le classi che mappano la struttura della base dati.
	
	Nel caso dell'applicazione web, queste sono le classi della \emph{Business Logic} che implementano i metodi setter, getter e i costruttori per creare l'oggetto mediante passaggio di dati di tipo primitivo od oggetti quali \texttt{RecordSet}.
	
	Si procede con l'elenco delle classi Entity nel package \texttt{BusinessLogic}:

\section{Diagrammi di sequenza}
Si riporta a titolo di esempio un Sequence Diagram relativo all'acquisto di un prodotto dello Store.

Le attività che coinvolgono i servizi bancari non sono state implementate nel progetto.